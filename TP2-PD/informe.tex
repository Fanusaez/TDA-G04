\section{Introducción}
El objetivo de este trabajo es ayudar a Scaloni a definir un plan de entrenamiento que provea la ganancia óptima.

Para lograr esto, se plantea un algoritmo de programación dinámica que obtenga el máximo beneficio obtenido por los entrenamientos, teniendo en cuenta la energía disponible desde el último descanso ($s_1, s_2, ...,s_n$) y el esfuerzo/ganancía de cada día $e_i$.

\subsection{Análisis del problema}
Resolver el problema a través de programación dinámica requiere plantear un caso base y una ecuación de recurrencia.
Primero hay que hallar sobre que variable/s se va a hacer inducción, en este caso resulta lógico hacer inducción sobre la cantidad de días $n$.

Pero solo con esta variable resulta imposible poder determinar para un día $i$ el óptimo solo teniendo en cuenta el óptimo de días anteriores.
Para poder plantear la ecuación de recurrencia, necesitamos una variable más, que por la forma en que se plantea la energía disponible, conviene que sea "días desde el último descanso"

Planteamos la siguiente proposición:

$P(n, d): $ Máximo beneficio en $n$ días de entrenamiento dado que el último descanso fue tomado hace $d$ días.

Luego el máximo beneficio para los $n$ días de entrenamiento es $\max_{0 \leq d \leq n} P(n, d)$